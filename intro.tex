
\begin{frame}
  \frametitle{\LaTeX\ 설치하기}
  \framesubtitle{}
  \begin{itemize}
    \item \TeX은 여러 가지 배포판이 있습니다.
    \item \TeX\,Live 배포판 설치하기 (가장 쉬운 방법)
    \begin{itemize}
      \item \url{https://www.tug.org/texlive/}에 방문하여 설치
      \item \texttt{texlive-full}을 받으면 정신건강에 매우 편하지만, 약 \textbf{4.9 GB}를 차지합니다.
      \item \texttt{texlive-basic}을 받으면 초기 용량을 \textbf{64 MB}까지 줄일 수 있지만, 필요한 패키지를 일일이 설치해야 합니다.
      \item 정식 release는 2018까지 있고, 현재 2019 preview가 진행되고 있습니다. \url{https://tug.org/texlive/pretest.html}
    \end{itemize}
    \item MikTeX (Windows) {\small(\url{https://miktex.org/})}
    \item TnXTeX {\small(\url{http://wiki.ktug.org/wiki/wiki.php/TnXTeX})}
    \item TinyTeX {\small(\url{https://yihui.name/tinytex/})}
  \end{itemize}
  \vskip.5em
  
  MikTeX은 \TeX\,Live 기반 배포판과 설정 방법이 달라 추천드리지 않습니다.
  
  \end{frame}
  
  \begin{frame}
  \frametitle{\TeX\,Live 패키지 설치하기 (스킵 가능)}
  \framesubtitle{}
  \begin{itemize}
    \item \texttt{texlive-full}이 아닌 더 적은 패키지를 포함하고 있는 버전을 받으셨다면, 추가로 설치해야 하는 패키지들이 있을 수 있습니다.
    \item 문서를 작성하다가 \texttt{File `\string~\string~\string~.sty' not found.}라는 에러 메시지가 발생하면, 다음 명령어를 터미널 (혹은 cmd)에 입력해 주세요.
    \begin{itemize}
      \item \texttt{tlmgr search --global --file "/패키지\_이름.sty"}를 입력하면
      \item `\texttt{패키지\_묶음\_이름:\quad texmf-dist/.../패키지\_이름.sty}'라는 문구를 볼 수 있고,
      \item \texttt{tlmgr install 패키지\_묶음\_이름}을 치면 설치됩니다.
    \end{itemize}
    \item 다운로드 받은 \texttt{sty} 파일을 설치하고 싶다면, \url{https://github.com/msquare-kaist/mathletter-package/blob/master/documents/manual.pdf}를 참조해 주세요.
  \end{itemize}
  
  \end{frame}
  
  
  \begin{frame}
    \frametitle{\TeX이란?}
    \framesubtitle{}
    \begin{itemize}
      \item \TeX: Donald Knuth가 개발한 문서 조판 도구
      \item \LaTeX: Leslie B. Lamport가 개발한 \TeX의 확장\,(매크로 모음)
      \begin{itemize}
        \item \LaTeXe: 가장 많이 쓰이는 \LaTeX\ 버전
        \item \LaTeX\,3: 현재 개발 중인 \LaTeX\ 버전
      \end{itemize}
      \item 이외에도 \ConTeXt이라는 확장이 있습니다.
      \item Overleaf: 가장 유명한 온라인 \LaTeX\ 동시 편집 서비스 (\url{https://overleaf.com})
    \end{itemize}
  \end{frame}
  
  \makeatletter
  \NewDocumentCommand\newterm{m o}{%
    \textbf{#1}%
    \IfNoValueF{#2}{\kern1pt{\small(#2)}\expandafter\ltx@ifnextchar@nospace{ }{}{\kern1pt}}%
  }
  \makeatother
  
  \begin{frame}[fragile]
    \frametitle{Token}
    \framesubtitle{}
    \begin{itemize}
      \item \TeX\ 문서는 토큰들로 이루어져 있으며, \TeX\ 엔진이 각 토큰을 \newterm{전개}[expand]하면서 실행됩니다.
      \item \newterm{매크로}[macro] 또는 \newterm{정의}[definition]란 기본적으로 정의되어 있는 토큰들{\,\small(primitives)\,}을 이용하여 정의된 토큰들을 말합니다.
      \item 매크로는 보통 \mintinline[escapeinside=||]{latex}{\macro}와 같이 백슬래시로 시작하고 알파벳이 뒤에 따라 옵니다.
      \item 한 글자짜리 매크로는 알파벳이 아니어도 올 수 있습니다. \mintinline[escapeinside=||]{latex}{\ }, \mintinline[escapeinside=||]{latex}{\#}, ...
      \item \{...\}는 여러 토큰을 하나처럼 묶어주는 토큰.
    \end{itemize}
  \end{frame}
  
  \begin{frame}[fragile]
    \frametitle{문서 구조}
    \framesubtitle{}
    \begin{itemize}
      \item \mintinline[escapeinside=||]{latex}{\documentclass[a4paper,11pt]{article}} 문서의 논리적 구조를 지정합니다.
      \begin{itemize}
        \item \texttt{book}, \texttt{report}, \texttt{memoir} 등의 class가 있습니다.
      \end{itemize}
      \item \mintinline[escapeinside=||]{latex}{\usepackage{amsmath}} 미리 정의된 매크로를 \newterm{패키지}에서 가져옵니다.
      \begin{itemize}
        \item 사실은 \mintinline[escapeinside=||]{latex}{\usepackage}는 \mintinline[escapeinside=||]{latex}{\RequirePackage}로 정의됩니다.
      \end{itemize}
      \item \mintinline[escapeinside=||]{latex}{\title{...}}, \mintinline[escapeinside=||]{latex}{\author{...}}, \mintinline[escapeinside=||]{latex}{\date{...}}
      \item \mintinline[escapeinside=||]{latex}{\begin{document}} 문서를 시작합니다.
      \item \mintinline[escapeinside=||]{latex}{\maketitle} 제목, 저자, 날짜를 출력합니다.
      \item \mintinline[escapeinside=||]{latex}{% 주석입니다} 퍼센트 기호 뒤는 무시됩니다.
      \item \mintinline[escapeinside=||]{latex}{\end{document}} 문서를 끝냅니다.
    \end{itemize}
  \end{frame}
  
  \begin{frame}[fragile]
    \frametitle{Sample Document}
    \framesubtitle{따라만 하세요 따라만 하세요}
    \vspace{-1.5em}
    \begin{minted}{latex}
\documentclass[a4paper,10pt]{article}
\usepackage{amsmath}
\usepackage{kotex}
\title{제목}

\makeatletter
\let\thetitle=\@title
\makeatother
\begin{document}
\maketitle

안녕하세요? \LaTeX을 배우면 좋은 일이 있을 거예요. % 아닐 수도..

당신이 입력했던 제목입니다: \thetitle.\footnote{신기한가요??}
\end{document}
  \end{minted}
  \end{frame}
  
  \begin{frame}[fragile]
    \frametitle{글씨체}
    \framesubtitle{인자를 받는 매크로와 스위치}
    \begin{itemize}
      \item 글씨체를 바꾸는 매크로는 크게 두 가지로 나뉩니다.
      \item 인자를 받는 매크로: \texttt{\{...\}} 안에 있는 글씨만 바꿉니다. \texttt{\char`\\text}나 \texttt{\char`\\math} 등으로 시작합니다.
      \begin{itemize}
        \item \mintinline[escapeinside=||]{latex}{\textit{italic here only} not here}
        \item \textit{italic here only} not here
      \end{itemize}
      \item 스위치: 어떤 scope 안에서, 그 매크로 이후의 글씨에 \textbf{모두 적용}됩니다. 사용에 주의해 주세요. \texttt{\char`\\...shape}나 \texttt{\char`\\...series}, \texttt{\char`\\...family}로 끝납니다.
      \begin{itemize}
        \item \mintinline[escapeinside=||]{latex}{\itshape{italic here and} also here}
        \item \itshape{italic here and} also here
      \end{itemize}
    \end{itemize}
  \end{frame}
  
  \begin{frame}[fragile]
    \frametitle{글씨체}
    \framesubtitle{일반 텍스트의 글씨체}
    \begin{itemize}
      \item \mintinline[escapeinside=||]{latex}{{\normalfont 기본 폰트}} {\normalfont 기본 폰트}
      \item \mintinline[escapeinside=||]{latex}{\textbf{굵게 bold}}, \mintinline[escapeinside=||]{latex}{{\bfseries 굵게 bold}} \textbf{굵게 bold}
      \item \mintinline[escapeinside=||]{latex}{\textit{이탤릭 italic}}, \mintinline[escapeinside=||]{latex}{{\itshape 이탤릭 italic}} \textit{이탤릭 italic}
      \begin{itemize}
        \item \mintinline[escapeinside=||]{latex}{\textit}는 그 다음에 똑바로 선 글자가 왔을 때 간격 보정을 해줍니다.
        \item \mintinline[escapeinside=||]{latex}{\textit{ital}ic}, \mintinline[escapeinside=||]{latex}{{\itshape ital}ic}
        \item \includegraphics[width=0.09\textwidth]{italic}
      \end{itemize}
      \item \mintinline[escapeinside=||]{latex}{\textrm{명조 roman}}, \mintinline[escapeinside=||]{latex}{{\rmfamily 명조 roman}} \textrm{명조 roman}
      \item \mintinline[escapeinside=||]{latex}{\textsf{돋움}}, \mintinline[escapeinside=||]{latex}{{\sffamily sans-serif}} \textsf{돋움 sans-serif}
      \item \mintinline[escapeinside=||]{latex}{\textsl{기울임}}, \mintinline[escapeinside=||]{latex}{{\slshape slanted}} \textsl{기울임 slanted}
      \item \mintinline[escapeinside=||]{latex}{\textsc{SmallCaps}}, \mintinline[escapeinside=||]{latex}{{\scshape SmallCaps}} \textsc{SmallCaps}
      \item \mintinline[escapeinside=||]{latex}{\texttt{typewriter}}, \mintinline[escapeinside=||]{latex}{{\ttfamily typewriter}} \texttt{typewriter}
    \end{itemize}
  \end{frame}
  
  \begin{frame}[fragile]
    \frametitle{글씨체}
    \framesubtitle{수식 모드의 글씨체}
    \begin{itemize}
      \item \mintinline[escapeinside=||]{latex}{\mathbf{Bold}} $\mathbf{Bold}$
      \item \mintinline[escapeinside=||]{latex}{\boldsymbol{\alpha}} $\boldsymbol{\alpha}$ (알파벳이나 숫자가 아닌 기호를 굵게)
      \item \mintinline[escapeinside=||]{latex}{\mathit{italic}} $\mathit{italic}$
      \item \mintinline[escapeinside=||]{latex}{\mathrm{Roman}} $\mathrm{Roman}$
      \item \mintinline[escapeinside=||]{latex}{\mathsf{Sans-serif}} $\mathsf{Sans-serif}$
      \item \mintinline[escapeinside=||]{latex}{\mathbb{BLACK}} $\mathbb{BLACK}$ (대문자만)
      \item \mintinline[escapeinside=||]{latex}{\mathcal{CALI}} $\mathcal{CALI}$ (대문자만)
      \item \mintinline[escapeinside=||]{latex}{\mathscr{SCRIPT}} $\mathscr{SCRIPT}$ (대문자만)
      \item \mintinline[escapeinside=||]{latex}{\mathfrak{Fraktur}} $\mathfrak{Fraktur}$
    \end{itemize}
  \end{frame}
  
  \begin{frame}[fragile]
    \frametitle{글씨 크기}
    \framesubtitle{}
    \begin{itemize}
      \item \mintinline[escapeinside=||]{latex}{\tiny \scriptsize \footnotesize \small \normalsize} \mintinline[escapeinside=||]{latex}{\large \Large \LARGE \huge \Huge}
      \item {\tiny A\scriptsize A\footnotesize A\small A\normalsize A\large A\Large A\LARGE A\huge A\Huge}
      \item \mintinline[escapeinside=||]{latex}{\fontsize{글씨 크기}{line height}\selectfont}
      \begin{itemize}
        \item 보통 line height는 글씨 크기의 1.2\,배
      \end{itemize}
      \item {\fontsize{60}{72}\selectfont A\fontsize{30}{36}\selectfont B\fontsize{50}{72}\selectfont C\fontsize{80}{86}\selectfont D}
    \end{itemize}
  \end{frame}
  
  \begin{frame}[fragile]
    \frametitle{수식 입력하기}
    \framesubtitle{}
    \begin{itemize}
      \item \mintinline[escapeinside=||]{latex}{$ ... $} 다른 글씨들과 같이 출력되는 수식입니다. 예) $\alpha^2\beta$
      \begin{itemize}
        \item \mintinline[escapeinside=||]{latex}{\( ... \)}을 쓸 수도 있지만, 이 매크로는 \textbf{fragile}한 매크로라 다른 매크로의 인자로 들어가면 오류를 내므로 사용하지 않는 편이 낫습니다.
      \end{itemize}
      \item \mintinline[escapeinside=||]{latex}{\[ ... \]} 한 줄 전체를 차지하는 수식입니다.
      \begin{itemize}
        \item \mintinline[escapeinside=||]{latex}{$$ ... $$}을 쓸 수도 있지만, 이 토큰은 여백 조절이나 수식 넘버링, 또 에러 메시지 출력 등에서 위의 것보다 안 좋습니다.
      \end{itemize}
      \item 기본적인 수식 매크로: \mintinline[escapeinside=||]{latex}{$ \alpha, \beta, \frac{\sqrt{2}}{3}, \cdots $}
      \begin{itemize}
        \item $ \alpha, \beta, \frac{\sqrt{2}}{3}, \cdots $
      \end{itemize}
      \item 더 많은 매크로와 기호는
      \begin{itemize}
        \item \url{https://www.codecogs.com/latex/eqneditor.php}나 
        \item \url{http://detexify.kirelabs.org/classify.html}에서!
      \end{itemize}
    \end{itemize}
  \end{frame}
  
  \begin{frame}[fragile]
    \frametitle{수식 매크로는 많이 외우셨으면 좋겠습니다}
    \begin{minted}{latex}
\documentclass[a4paper,10pt]{article}
\usepackage{amsmath,amsfonts,amssymb,mathrsfs,mathtools,kotex}
\title{\scshape Practice}
\begin{document}
\maketitle
\[ \text{근의 공식}:\qquad x = \frac{-b\pm\sqrt{b^2 - 4ac}}{2a}. \]
$\alpha, \beta$는 있지만 \texttt{\char`\\Alpha, \char`\\Beta}는 없다!

Radical of an ideal $\sqrt{\mathfrak a}$

크기가 조절되는 괄호를 열 때에는 $\left( \frac a b \right)$처럼 씁니다.
강의할 때 어느 정도 설명해주세요 \char`\^\char`\^7
\end{document}
    \end{minted}
    \char`\^\char`\^7
  \end{frame}


  \begin{frame}[fragile]
    \frametitle{공백, 행간 간격}
    \framesubtitle{}
    \begin{itemize}
      \item<1-> 문단을 띄울 때에는 엔터를 두 번 치거나 \mintinline[escapeinside=||]{latex}{\par}를 입력합니다.
      \item<1-> 강제로 한 줄을 띄울 때에는 \mintinline[escapeinside=||]{latex}{\\}를 입력합니다.
      \item<2-> \mintinline[escapeinside=||]{latex}{\! \, \ \quad \qquad ~}는 텍스트 모드에서의 공백을 나타냅니다. 크기는 다음과 같습니다.
      \begin{itemize}
        \item \mintinline[escapeinside=||]{latex}{a a\!a\,a\ a\quad a\qquad a~a} $\to$ a a\!a\,a\ a\quad a\qquad a~a
        \item \mintinline[escapeinside=||]{latex}{\!}는 간격을 줄입니다.
        \item \mintinline[escapeinside=||]{latex}{~}는 끊어지지 않는 공백을 나타내는 `active character'입니다.
      \end{itemize}
      \item<2-> \mintinline[escapeinside=||]{latex}{\! \, \ \> \: \; \quad \qquad ~}는 수식 모드에서의 공백을 나타냅니다. (직접 테스트 해보세요)
      \item<3-> \mintinline[escapeinside=||]{latex}{\hspace{길이}}, \mintinline[escapeinside=||]{latex}{\hspace*{길이}}는 가로로 $\langle$길이$\rangle$만큼의 간격을 만듭니다. 원래는 한 줄을 시작할 때엔 무시되는데, *이 붙으면 살아납니다.
      \item<3-> \mintinline[escapeinside=||]{latex}{\vspace{길이}}, \mintinline[escapeinside=||]{latex}{\vspace*{길이}}는 세로 버전
      \item<4-> \mintinline[escapeinside=||]{latex}{\noindent}는 문단 시작 시 들여쓰기를 없앱니다.
    \end{itemize}
  \end{frame}


  \begin{frame}[fragile]
    \frametitle{공백 관련}
    \framesubtitle{}
    \begin{itemize}
      \item<1-> \TeX은 띄어쓰기에 민감합니다. 예를 들어 괄호 앞뒤로 띄어쓰기를 하는지가 상관없는 C나 Python 등 다른 언어와 달리, \TeX\ 문서를 쓸 때엔 띄어쓰기를 주의해야 합니다.
      \item<2-> 텍스트 모드에서 띄어쓰기가 없어지는 경우는:
      \begin{itemize}
        \item 두 개 이상 연속 띄어쓰기가 있을 때 한 개만 남습니다. 즉 띄어쓰기 여러 개를 쓰려면 \mintinline{latex}{\ \ \ }와 같이 써야 합니다.
        \item 한 줄의 맨 앞{\,\small(leading)\normalsize\,}에 있는 띄어쓰기는 모두 무시됩니다.
        \item 매크로 뒤의 공백은 (특별한 처리를 하지 않은 이상) 무시됩니다. \mintinline{latex}{\TeX \TeX} $\to$ \TeX \TeX
      \end{itemize}
      \item<3-> 텍스트 모드에서, 줄바꿈의 위아래 줄이 비어있지 않으면 그 줄바꿈은 띄어쓰기로 처리됩니다.
      \begin{itemize}
        \item 이 띄어쓰기를 없애려면 그 줄 끝에 주석 기호 \texttt{\%}를 붙이면 그 줄바꿈이 주석 처리되어 사라집니다. 참고로, 다음 줄의 앞에 있는 띄어쓰기까지 모두 사라집니다.
      \end{itemize}
    \end{itemize}
  \end{frame}


  \begin{frame}[fragile]
    \frametitle{공백 관련}
    \framesubtitle{}
    \begin{itemize}
      \item 수식 모드에서는 띄어쓰기가 잘 영향을 주지 않습니다. 따라서 수식 안에서 텍스트 모드를 실행하려면, \mintinline{latex}{\text{...}} 등으로 감싸 주어야 합니다.
      \begin{itemize}
        \item \mintinline{latex}{$not a good example$} $not a good example$
        \item \mintinline{latex}{$\textrm{\textit{good example}}$} $\textrm{\textit{good example}}$
      \end{itemize}
    \end{itemize}
  \end{frame}
    
  
  \begin{frame}[fragile]
    \frametitle{달러나 백슬래시는 어떻게 입력할까요}
    \framesubtitle{미국인이 만들었는데 달러를 입력 못할 리가 없겠죠!!}
    \begin{itemize}
      \item<1-> \mintinline[escapeinside=||]{latex}{\& \% \$ \# \_ \{ \} \char`~ \|\textvisiblespace| \char`^ \|\textvisiblespace| \char`\\} 
      \item<1-> \& \% \$ \# \_ \{ \} \char`\~\ \char`\^\ \char`\\
      \item<2-> 살짝씩 다르지만 다른 입력 방법도 많습니다:
      \item<2-> \mintinline[escapeinside=||]{latex}{\textasciitilde\|\textvisiblespace|\textasciicircum\|\textvisiblespace|\textbackslash}
      \item<2-> \textasciitilde\ \textasciicircum\ \textbackslash
      \item<3-> \mintinline[escapeinside=||]{latex}{\^{} \~{} \string^ \string~ \string\}
      \item<3-> \^{} \~{} \string^ \string~ \string\
    \end{itemize}
  \end{frame}
  