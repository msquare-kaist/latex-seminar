
\begin{frame}[fragile]
  \frametitle{조금 더 복잡한 예시}
  \framesubtitle{이 정도는 익혀두면 과제할 때 손으로 계산할 필요가 없습니다}
  \begin{itemize}
    \item<1-> 유클리드 알고리듬을 구현해봅시다! 나누어질 수(\texttt{\#1})와 나눌 수(\texttt{\#2})를 받아 표를 만듭니다. 편의를 위해 $\texttt{\#1}\ge\texttt{\#2} > 0$이라고 가정합니다.
    \item<2-> 먼저 전체를 감쌀 array가 필요하므로, array로 감싸고 그 안에 있는 내용은 다른 매크로에게 맡깁니다.
    \item[]<2-> \vspace{-1.5em}\begin{minted}{latex}
\makeatletter\ExplSyntaxOn
\begin{array}{r@{\;=\;}r@{\;\cdot\;}l@{}l
              @{}l @{\quad}|@{\quad} r@{}r@{}r@{}r@{}r}
  ... % 다른 매크로
\end{array}
\makeatother\ExplSyntaxOff
    \end{minted}
    \item<3-> 생각해 보면 나누어질 수와 나눌 수, 몫, 나머지, 이전 두 단계의 결과를 저장해야 되니까 10 개의 argument를 사용해야 하는데, 이것은 불가능합니다. 따라서 나누어질 수와 나눌 수는 전역변수로 저장하도록 합시다.
    \item[]<3-> \mintinline{latex}{\gdef\@@a{#1}\gdef\@@b{#2}}
  \end{itemize}
\end{frame}

\begin{frame}[fragile]
  \frametitle{조금 더 복잡한 예시}
  \framesubtitle{이 정도는 익혀두면 과제할 때 손으로 계산할 필요가 없습니다}
  \begin{itemize}
    \item<1-> 그리고 안에 들어갈 매크로에 데이터를 전달합니다.
    \item[]<1-> \vspace{-1.5em}\begin{minted}{latex}
...\begin{array}{r@{\;=\;}r@{\;\cdot\;}l@{}l
              @{}l @{\quad}|@{\quad} r@{}r@{}r@{}r@{}r}
  \euclid@lgorithm               % initialize
  {#1}{#2}                       % a and b
  {\int_div_truncate:nn{#1}{#2}} % quotient q = floor(a / b)
  {\int_mod:nn{#1}{#2}}          % remainder r = a mod b
  {0}{1}                         % b = 0 * a + 1 * b
  {1}{\the\numexpr
      -\int_div_truncate:nn{#1}{#2}
                            \relax} % r = 1 * a + (-q) * b
\end{array}...
    \end{minted}
  \end{itemize}
\end{frame}

\begin{frame}[fragile]
  \frametitle{조금 더 복잡한 예시}
  \framesubtitle{이 정도는 익혀두면 과제할 때 손으로 계산할 필요가 없습니다}
  \begin{itemize}
    \item<1-> 이제 안쪽에 들어갈 매크로를 만듭니다. 나머지가 0일 때와 아닐 때로 나눕시다.
    \item<1-> 나머지가 0일 때
    \item[]<1-> \vspace{-1.5em}\begin{minted}{latex}
\newcommand{\euclid@lgorithm}[8]{
  \int_compare:nNnTF {#4} = {0}
  {
    % if remainder is 0, terminate the recursion
    #1 & \mathbf{#2} & #3 &&&
  }
  { ... }
}
    \end{minted}
  \end{itemize}
\end{frame}

\begin{frame}[fragile]
  \frametitle{조금 더 복잡한 예시}
  \framesubtitle{이 정도는 익혀두면 과제할 때 손으로 계산할 필요가 없습니다}
  \begin{itemize}
    \item<1-> 나머지가 0이 아닐 때
    \item[]<1-> \vspace{-1.5em}\begin{minted}{latex}
\newcommand{\euclid@lgorithm}[8]{
  \int_compare:nNnTF {#4} = {0} { ... } {
    % print a line
    #1 & #2 & #3 \;&\;+\;&\; #4 & #4 &\;=\;&\;\@@a\;\cdot\; &
    \,{\ifnum#7<\z@ (#7) \else #7\hphantom{)}\fi}+\@@b\;\cdot\;&
    \,{\ifnum#8<\z@ (#8) \else #8\hphantom{)}\fi} \\%
    % call itself recursively
    \euclid@lgorithm
    {#2}{#4}
    {\int_div_truncate:nn{#2}{#4}}
    {\int_mod:nn{#2}{#4}}
    {#7}{#8}
    {\the\numexpr #5 - \int_div_truncate:nn{#2}{#4} * #7\relax}
    {\the\numexpr #6 - \int_div_truncate:nn{#2}{#4} * #8\relax}% 
  }
}
    \end{minted}
  \end{itemize}
\end{frame}
\makeatletter
\def\eqv#1#2{\let\mod=\m@d{#1}\expandafter\eqvB{#2}}
\def\eqvB#1{
    \@ifnextchar\mod{\equiv#1}{\equiv#1\expandafter\eqvB}
}
\def\m@d#1{\quad({\operatorname{mod}}\;#1)}
\makeatother
\setstretch{1.3}
\begin{frame}[fragile]
  \frametitle{배열을 다루는 법}
  \framesubtitle{plain\,\TeX에서는 (첫 번째 방법)}
  \begin{itemize}
    \item Expand가 되지 않는 토큰을 넣어서 delimiter의 역할을 하게 합니다.
    \item 예를 들어서 여러 식이 합동인 상황 $\eqv{F_1}{F_2}{F_3}\mod{N}$을 나타내고 싶을 때 \mintinline{latex}{\eqv{F_1}{F_2}{F_3}\mod{N}}이라고 쓰면 편할 것입니다. 이런 상황은 다음과 같이 구현할 수 있습니다.
    \vspace{-1.5em}\begin{minted}{latex}
\makeatletter
\def\eqv#1#2{\let\mod=\m@d{#1}\expandafter\eqvB{#2}}
\def\eqvB#1{
  \@ifnextchar\mod{\equiv#1}{\equiv#1\expandafter\eqvB}
}
\def\m@d#1{\quad({\operatorname{mod}}\;#1)}
\makeatother
    \end{minted}
  \end{itemize}
\end{frame}
\begin{frame}[fragile]
  \frametitle{배열을 다루는 법}
  \framesubtitle{plain\,\TeX에서는 (두 번째 방법)}
  \begin{itemize}
    \item 매크로 정의에 delimiter를 명시적으로 넣습니다. (패턴 매칭 기능)
    \item 토큰 리스트를 뒤집는 매크로를 만들어봅시다.
    \vspace{-1.5em}\begin{minted}{latex}
\def\firstoftwo#1#2{#1}
\def\secondoftwo#1#2{#2}

\def\rev#1#2\revA#3\revB{%
  \if\relax\detokenize{#2}\relax
    \expandafter\firstoftwo
  \else
    \expandafter\secondoftwo
  \fi{#1#3}{\rev#2\revA#1#3\revB}%
}
    \end{minted}
  \end{itemize}
\end{frame}

\begin{frame}[fragile]
  \frametitle{배열을 다루는 법}
  \framesubtitle{plain\,\TeX에서는 (세 번째 방법)}
  \begin{itemize}
    \item 배열 원소를 각각 다른 매크로로 정의합니다. \mintinline{latex}{\csname NAME \endcsname}은 \mintinline{latex}{\NAME}과 같은데, \texttt{NAME}이 알파벳만으로만 이루어져 있지 않아도 정의됩니다. (아래 예제에서는 카운터를 \LaTeX\ 카운터를 썼습니다.)
    \vspace{-1.5em}\begin{minted}{latex}
\makeatletter
\expandafter\def\csname my@array@1\endcsname{1}
\expandafter\def\csname my@array@2\endcsname{2}

% with counter
\newcounter{arrayindex}
\setcounter{arrayindex}{3}
\expandafter\def\csname my@array@\value{arrayindex}\endcsname
  {\value{arrayindex}}
\makeatother
    \end{minted}
  \end{itemize}
\end{frame}

\begin{frame}[fragile]
  \frametitle{배열을 다루는 법}
  \framesubtitle{\LaTeX\,3에서는}
  \begin{itemize}
    \item \LaTeX\,3에는 토큰 리스트를 위한 함수들이 따로 마련되어 있습니다. \mintinline{latex}{tl}이 포함된 매크로들이 그것입니다.
    \item 그런데 제가 어떻게 쓰는지 몰라서\dots
    \item \url{https://www.texdev.net/2011/12/26/programming-latex3-token-list-variables/}
  \end{itemize}
\end{frame}