
\begin{frame}[fragile]
  \frametitle{매크로}
  \framesubtitle{귀찮은 일은 싫어하는 사람들의 모임}
  \begin{itemize}
    \item<1-> 편집부장님: ``레이텍으로 구구단을 만들어 주세요!''
    \item<1-> 안타까운 사람들: \mintinline[escapeinside=||]{latex}{$2 \times 2 = 4$, $2 \times 3 = 6$, ...}
    \item<2-> 매크로를 배운 후:\vspace{-1em}
    \begin{minted}{latex}
\newcounter{left} \newcounter{right}
\forloop{left}{2}{\value{left} < 10}{%
  \noindent\forloop{right}{2}{\value{right} < 10}{%
    $\arabic{left} \times \arabic{right} =
      \the\numexpr\arabic{left} * \arabic{right}\relax$,
  }
  \par
}
    \end{minted}
  \end{itemize}
\end{frame}

\begin{frame}[fragile]
  \frametitle{기본적인 매크로}
  \framesubtitle{plain \TeX 버전}
  \begin{itemize}
    \item \mintinline[escapeinside=||]{latex}{\def\mymacro#1#2#3{\textit{#1}#2\textrm{#3}}}
    \begin{itemize}
      \item \mintinline[escapeinside=||]{latex}{\def} 매크로 정의를 시작하는 토큰
      \item \mintinline[escapeinside=||]{latex}{\mymacro#1#2#3} 매크로 이름과 parameter 개수 (<9), 즉 parameter를 \mintinline[escapeinside=||]{latex}{\mymacro{p1}{p2}{p3}}처럼 받는다는 이야기
      \item \mintinline[escapeinside=||]{latex}{{\textit{#1}#2\textrm{#3}}} 매크로 이름이 나온 후 처음으로 \texttt{\{}를 만나면 그 매크로를 전개했을 때 나오는 매크로의 `내용'을 정의하게 됩니다.
    \end{itemize}
    \item 다른 기호를 쓸 수도 있어요!! \mintinline[escapeinside=||]{latex}{\mymacro[#1]#2\middle #3{#1#2#3}} 패턴 매칭처럼 행동합니다. 이 방법은 plain \TeX에서 가끔 배열을 다뤄야 할 때 유용해요. 
    \item 한글은 쓰지 마세요...
  \end{itemize}
\end{frame}
\begin{frame}[fragile]
  \frametitle{기본적인 매크로}
  \framesubtitle{\LaTeXe 버전}
  \begin{itemize}
    \item \mintinline[escapeinside=||]{latex}{\newcommand{\mymacro}[3][A]{#1,#2,#3}} 
    \begin{itemize}
      \item \mintinline[escapeinside=||]{latex}{\newcommand} 매크로 정의를 시작하는 토큰
      \item \mintinline[escapeinside=||]{latex}{{\mymacro}} 매크로 이름
      \item \mintinline[escapeinside=||]{latex}{[3]} parameter 개수 (<9)
      \item \mintinline[escapeinside=||]{latex}{[A]} \texttt{\#1}의 기본값 (기본값은 가장 처음 인자 한 개까지만 지정 가능하며, 기본값이 있는 optional한 인자는 \texttt{\{...\}}가 아닌 \texttt{[...]}로 묶어주어야 합니다.)
    \end{itemize}
  \end{itemize}
\end{frame}
\begin{frame}[fragile]
  \frametitle{기본적인 매크로}
  \framesubtitle{\LaTeX\,3 버전}
  \begin{itemize}
    \item \mintinline[escapeinside=||]{latex}{\NewDocumentCommand{\mymacro}{s m o O{3}}{#1,#2,#3,#4}} 
    \begin{itemize}
      \item \mintinline[escapeinside=||]{latex}{\NewDocumentCommand} 매크로 정의를 시작하는 토큰
      \item \mintinline[escapeinside=||]{latex}{{\mymacro}} 매크로 이름
      \item \mintinline[escapeinside=||]{latex}{{s m o O{3}}} parameter 개수와 그 spec
      \begin{itemize}
        \item \texttt{m}: 필수 인자
        \item \texttt{o}: 기본값 없는 optional한 인자; 기본값 없는 optional 인자에 값이 주어지지 않으면 대신 \texttt{-NoValue-}라는 값이 들어갑니다.
        \item \texttt{O\{3\}}: 기본값이 3인 optional한 인자
        \item \texttt{s}: 별표가 있는지 없는지 감지
        \item 다른 spec은 패키지 \texttt{xparse} 매뉴얼 참조
      \end{itemize}
    \end{itemize}
    \item 지금까지 나온 매크로들은 텍스트와 수식 모드 모두에서 쓸 수 있습니다.
  \end{itemize}
\end{frame}

\begin{frame}[fragile]
  \frametitle{매크로 예시!!}
  \framesubtitle{간단하고 쉬운 것}
  \begin{itemize}
    \item $\boldsymbol{\alpha}$를 간단하게 쓰고 싶을 때:
    \item \mintinline[escapeinside=||]{latex}{\newcommand{\bsalpha}{\boldsymbol{\alpha}}} \newcommand{\bsalpha}{$\boldsymbol{\alpha}$} \bsalpha
    \item $\mathbb Z$를 간단하게 쓰고 싶을 때:
    \item \mintinline[escapeinside=||]{latex}{\newcommand{\ZZ}{\mathbb{Z}}} \newcommand{\ZZ}{{\mathbb Z}} $\ZZ$
  \end{itemize}
\end{frame}

\begin{frame}[fragile]
  \frametitle{매크로 예시!!}
  \framesubtitle{간단하고 쉬운 것}
  \begin{itemize}
    \item 벡터 표기하기
    \begin{itemize}
      \item \mintinline[escapeinside=||]{latex}{$\vecNotation{X}$}를 하면 $(X_1, X_2, X_3)^{\mathsf{T}}$가 나오게 하고 싶다!
      \item \mintinline[escapeinside=||]{latex}{$\vecNotation{X}[1]$}를 하면 $(X_1)$가 나오게 하고 싶다!
      \item \mintinline[escapeinside=||]{latex}{$\vecNotation{X}[2]$}를 하면 $(X_1, X_2)^{\mathsf{T}}$가 나오게 하고 싶다!
      \item \mintinline[escapeinside=||]{latex}{$\vecNotation{X}[4]$}를 하면 $(X_1, X_2,\dots,X_4)^{\mathsf{T}}$가 나오게 하고 싶다!
      \item \mintinline[escapeinside=||]{latex}{$\vecNotation{X}[n]$}를 하면 $(X_1, X_2,\dots,X_n)^{\mathsf{T}}$가 나오게 하고 싶다!
    \end{itemize}
  \end{itemize}
\end{frame}

\setstretch{0.9}
\begin{frame}[fragile]
  \frametitle{매크로 예시!!}
  \framesubtitle{간단하고 쉬운 것}
  \vspace*{-2em}
  \begin{minted}[fontsize=\footnotesize]{latex}
\ExplSyntaxOn % _와 :를 매크로 이름에 쓸 수 있게 만듭니다.
\NewDocumentCommand{\vecNotation}{m O{3}}{
  % \tl_if_eq:nnTF{토큰 리스트1}{토큰 리스트2}{같을 때}{다를 때}
  \tl_if_eq:nnTF{#2}{1}{
    (#1\sb 1) % \ExplSyntaxOn이 _를 문자로 바꾸므로 
              % \sb라는 명령어로 subscript를 만듭니다.
  }{
    \tl_if_eq:nnTF{#2}{2}{
      (#1\sb 1, #1\sb 2)^{\mathsf{T}}
    }{
      \tl_if_eq:nnTF{#2}{3}{
        (#1\sb 1, #1\sb 2, #1\sb 3)^{\mathsf{T}}
      }{
        (#1\sb 1, #1\sb 2, \dots, #1\sb{#2})^{\mathsf{T}}
      }
    }
  }
}
\ExplSyntaxOff % _와 :를 다시 각자의 목적으로 되돌립니다.
  \end{minted}
\end{frame}
\setstretch{1.3}


\begin{frame}[fragile]
  \frametitle{Counter}
  \framesubtitle{For \LaTeXe}
  \begin{itemize}
    \item 정수(integer)형 변수 역할
    \item \mintinline{latex}{\newcounter{CounterName}} 새 카운터를 만듭니다.
    \item \mintinline{latex}{\setcounter{CounterName}{value}} 카운터 값 지정
    \item \mintinline{latex}{\addtocounter{CounterName}{value}} 카운터에 값을 더하기 (음수도 가능)
    \item \mintinline{latex}{\value{CounterName}} 카운터 값 출력
  \end{itemize}
\end{frame}




\begin{frame}[fragile]
  \frametitle{For loop}
  \framesubtitle{\texttt{forloop}, \texttt{pgffor}}
  \begin{itemize}
    \item \mintinline{latex}{\forloop[step]{counter}{initial}{condition}{code}}
    \begin{itemize}
      \item \mintinline{latex}{\usepackage{forloop}}
      \item \mintinline{latex}{\newcounter{ct}}
      \item[] \mintinline{latex}{\forloop{ct}{1}{\value{ct} < 10}{\arabic{ct} }}
      \item 1 2 3 4 5 6 7 8 9\textvisiblespace
    \end{itemize}
    \item \mintinline{latex}{\foreach \macro in {1,4,...,10}{...}}
    \begin{itemize}
      \item \mintinline{latex}{\usepackage{pgffor}}
    \setstretch{1.1}
    \item[\mysymboli\;] \vspace{-1.5em}\begin{minted}[fontsize=\footnotesize]{latex}
\newcommand{\repsum}[3]{%
  \foreach \i in {1,...,#1}{
    \ifnum\i>1
      + #2_{\i} #3_{\i}
    \else
      #2_{\i} #3_{\i}
    \fi
  }
}
      \end{minted}
    \setstretch{1.3}
  \end{itemize}
  \end{itemize}
\end{frame}


\begin{frame}[fragile]
  \frametitle{커버하지 않은 내용}
  \framesubtitle{너무 많아요}
  \texttt{\char`\\label\char`\\ref\char`\\url\char`\\definecolor\char`\\textcolor\char`\\color\char`\\newenvironment\\\char`\\expandafter\char`\\noexpand\char`\\IfNoValueTF\char`\\IfBooleanTF\char`\\makeatletter\\\char`\\makeatother\char`\\@author\char`\\@date\char`\\begin\{tikz\}...\char`\\end\{tikz\}\char`\\edef\\\char`\\@gobble\char`\\relax\char`\\dimexpr\char`\\numexpr
  \char`\\the\char`\\meaning\char`\\global\\\char`\\xdef\char`\\csname...\char`\\endcsname\char`\\@firstofone\char`\\z@\char`\\@car\char`\\@cdr\char`\\@nil\\\char`\\toks@\char`\\loop\char`\\@for\char`\\g@addto@macro\char`\\parskip\char`\\parindent\char`\\protected\\ \char`\\unexpanded\char`\\renewcommand\char`\\RedeclareSectionCommand\char`\\hbox\char`\\pbox\\\char`\\rule\char`\\raisebox\char`\\kern\char`\\hskip\char`\\vskip\char`\\glueexpr\char`\\allowdisplaybreaks,...}

  \vspace{1em}
  패키지: \texttt{expl3}, \texttt{xkeyval}, \texttt{pgf}, \texttt{tikz}, \texttt{tikzcd}, \texttt{ulem}, \texttt{tcolorbox}, \texttt{bibtex}, \texttt{natbib}, \texttt{fancyhdr}, \texttt{tcolorbox}, \texttt{environ}, \texttt{listings}, \texttt{minted}, \texttt{beamer, \dots}

  \vspace{1em} 다른 언어의 도움: \LuaLaTeX, HaTeX, PyTeX, \dots
\end{frame}